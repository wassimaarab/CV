

\documentclass{ExpressiveResume}
\usepackage{graphicx}

\begin{document}

\noindent
\hspace{1in}
\begin{minipage}[c]{0.10\textwidth}
    \includegraphics[width=\linewidth,clip,trim=0 0 0 0]{images/Me and myself.jpg} % <-- replace with your image file
\end{minipage}
\hfill
\begin{minipage}[c]{0.70\textwidth}
    \resumeheader[
        firstname=WASSIM,
        lastname=AARAB,
        email=1wassimaarab@gmail.com,
        phone=+33 748534953,
        linkedin=wassim-aarab-646b72283,
        github=,
        city=,
        state=,
        qrcode=,
    ]
\end{minipage}

Étudiant en \textbf{3ème année} en informatique à
 l'école d'ingénieurs \textbf{ENSEIRB-MATMECA} option \textbf{Intelligence Artificielle}, recherchant un stage d'au moins \textbf{5 mois} à partir du \textbf{2 février}.

\section{Parcours académique}
\degree{ENSEIRB-MATMECA, Bordeaux-France}{Ingénieur en informatique option IA}{2023- 2026}{
    \achievement{
        Cours pertinents : Machine Learning, Deep Learning, Data Science,
Algorithmes numériques, Théorie des graphes, Probabilités et statistiques,
Cryptologie, Architecture des réseaux, Programmation.
    }
}
\degree{Lycée Moulay Driss, Fes-Maroc}{Classe préparatoires MP}{2021- 2023}{

}


\section{Expérience professionnelle}
\project{Stage 2A [4mois] - French Touch Travel}{Juin - Octobre 2025}{
Développement d’un \textbf{chatbot intelligent} intégré à \textbf{Odoo}, utilisant des \textbf{LLM} et \textbf{agents d’IA} orchestrés via \textbf{N8n}. Automatisation des tâches de vente : transcription (\textbf{Whisper}), diarisation (\textbf{Pyannote}), recherche d’informations et génération d’itinéraires via \textbf{RAG}.}
%\vspace{0.10in}
\newline
\project{Convention Citoyenne inter-universitaire sur l’IA}{Avril 2025}{
Participation à un débat sur l’éthique de l’IA, la sécurité des données et la durabilité numérique.
}
%\vspace{0.10in}
\newline
\project{Stage 1A [2mois] - C-MonEtiquette}{Juin - Août 2024}{
Programmation et maintenance d’une machine de découpe industrielle au sein du pôle Impression.
}

\section{Projets académique}
\project{Génération de playlists musicales intelligentes}{Sep. - Nov. 2025}{
Implémentation d’un \textbf{CNN} inspiré de van den Oord pour générer des playlists musicales par similarité (genre, émotions ...).}
%\vspace{0.10in}
\newline
\project{Colorisation automatique d’images en niveaux de gris}{Sep. - Nov. 2025}{
Modèle de \textbf{deep learning} prédisant les canaux de couleur manquants pour des images en niveaux de gris (Auto-encodeurs, CNN, GAN, colorisation guidée, marqueurs utilisateurs).}%\vspace{0.10in}
\newline
\project{PFA - Zeste de savoir}{Mars. - Août. 2025}{
Intégration d’un module \textbf{anti-spam}, mise à jour du système JS et amélioration de la gestion du contenu.(TensorFlow, ORM Django)}
%\vspace{0.10in}
\newline
\project{Joeur de GO}{Jan. - Mars 2025}{
Développement d’un agent IA basé sur \textbf{Alpha–Beta} et \textbf{CNN} pour la prédiction de coups.}

\section{Compétences techniques}
\textbf{Machine \& Deep Learning :} KNN, SVM, Random Forest, CNN, LSTM, Transformers, Auto-encodeurs, RAG, fine-tuning, TensorFlow.\\
\textbf{Langages :} Python(\textit{PyTorch, Scikit-learn, Pandas, numpy}), Java, C++, C, JavaScript, SQL.\\
\textbf{Cloud \& Modélisation :} OCI (AI/ML, Generative AI), Azure, UML.\\
\textbf{DevOps :} Docker, CI/CD (\textit{GitHub Actions, GitLab, Jenkins, ArgoCD}), Grafana, Pulumi, Vercel.\\
\textbf{Web :} Java (Spring Boot), JavaScript, TypeScript, REST API.

\section{Langues}
\textbf{Français :} Courant \hfill
\textbf{Anglais :} C1 (IELTS) \hfill
\textbf{Espagnol :} A2

\section{Centres d’intérêt}
\textbf{Sport :} Calisthenics, course (400–800 m), football \hfil
\textbf{Autres :} Écriture, prise de parole en public


\end{document}